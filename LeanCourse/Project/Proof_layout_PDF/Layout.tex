\documentclass{../Proof_layout_PDF/TemplateExercise}

\renewcommand{\SheetName}{Project overview}
\renewcommand{\LectureName}{Lean}
\renewcommand{\mName}{Ludwig Monnerjahn}

\begin{document}


\section{The TemplateExercise Problems}
\textbf{Doubling the cube}, also known as the Delian problem, is an ancient geometric problem.
Given the edge of a cube, the problem requires the construction of the edge of a second cube whose volume is double that of the first,
using only a ruler and compass.
\newline
\textbf{Angle trisection} is the construction, using only a ruler and compass, of an angle that is one third of a given arbitrary angle.

\subsection{Definitions}
First we need to define what construction using a ruler and compass means.
We will use $\IC$ as plane of drawing and $\mathcal{M} \subset \IC$ as the set of constructed points.
\begin{definition}
    $\mathcal{G(M)}$ is the set of all real straight lines $\mathcal{G}$, with $| \mathcal{G} \cap \mathcal{M} |\ge 2$.\newline
    $\mathcal{C(M)}$ is the set of all circles in $\IC$, with center in $\mathcal{M}$ and radius of $\mathcal{C}$ is the distence of two points in $\mathcal{M}$.
\end{definition}

\begin{definition}
    We define operation that can be used to constructed new Points.
    \begin{enumerate}
        \item $(ZL 1)$ is the cut of two lines in $\mathcal{G(M)}$.
        \item $(ZL 2)$ is the cut of a line in $\mathcal{G(M)}$ and a circle in $\mathcal{C(M)}$.
        \item $(ZL 3)$ is the cut of two circles in $\mathcal{C(M)}$.
    \end{enumerate}
    $ZL(\mathcal{M})$ is the set $\mathcal{M}$ combeined with of all points that can be constructed using the operations $(ZL 1)$, $(ZL 2)$ and $(ZL 3)$.
\end{definition}

\begin{definition}
    We define inductively the the chain
    \begin{equation*}
        \mathcal{M}_0 \subseteq \mathcal{M}_1 \subseteq \mathcal{M}_2 \subseteq \dots
    \end{equation*}
    with $\mathcal{M}_0 = \mathcal{M}$ and $\mathcal{M}_{n+1} = ZL(\mathcal{M}_n)$.\newline
    And call $\mathcal{M}_{\infty} = \bigcup_{n \in \IN} \mathcal{M}_n$ the set of all constructable points.
\end{definition}


\subsection{Problem simplification}
Let $\mathcal{M} = \{a, b\}$. Let $r := \|a - b\|$ be the distance between $a$ and $b$. Then a Qube with edge $r$ has volume $r^3$.
There is a cube with volume $2r^3$ if and only if $\sqrt[3]{2} \in \mathcal{M}_{\infty}$.
\begin{problem}
    Let $\mathcal{M} = \{0,1\}$. Is $\sqrt[3]{2} \in \mathcal{M}_{\infty}$?
\end{problem}
Let $\mathcal{M} = \{a, b, c\}$ with $a, b, c$ not on a line. Let $\alpha := \angle (b - a, c - a) $.
Then $\alpha$ can be trisected if and only if, there is a point $d\in \mathcal{M}_{\infty}$ so that $\angle (b - a, d - a) = \alpha/3$. Using a
"standard" $\mathcal{M} = \{0,1,\exp(\textbf{i} \alpha)\}$ gives us the following problem.
\begin{problem}
    Let $\mathcal{M} = \{0,1,\exp(\textbf{i} \alpha)\}$. Is $\exp(\textbf{i} \alpha/3) \in \mathcal{M}_{\infty}$?
\end{problem}

\subsection{Properties of the the set of constructable points}
\begin{definition}
    The degree of $x$ over $K$ is
    \begin{equation*}
        [x:K] :=\text{degree}(\mu_{x,K})
        \end{equation*}
        with $\mu_{x,K}$ the minimal polynomial of $x$ over $K$. \newline
    The degree of $L/K$ is the dimension of $L$ as a $K$-vector space and is denoted by
    \begin{equation*}
        [L:K].
    \end{equation*}
\end{definition}

\begin{theorem}
\label{thm:degree_of_simple_field_extension}
    Let $L/K$ be a simple field extension with $L = K(x)$. Then
    \begin{equation*}
        [L:K] = [x:K].
    \end{equation*}
\end{theorem}
\begin{proof}
    In Mathlib: theorem IntermediateField.adjoin.finrank
\end{proof}
\begin{definition}
    Let $\mathcal(M)\subseteq\IC$ with $0,1 \in \mathcal{M}$
    \begin{equation*}
        K_0 := \IQ(\mathcal{M}\cup \overline{\mathcal{M}})
    \end{equation*}
    with $\overline{\mathcal{M}} := \{ \overline{z} = x - \textbf{i}y \mid z = x+\textbf{i}y  \in \mathcal{M} \}$.
\end{definition}
%\begin{remark}
%    We know that $K_0$ is a field and $\mathcal{M} \subseteq K_0\subseteq \mathcal{M}_{\infty} \subseteq \IC$.
%\end{remark}
\begin{theorem}
\label{thm:degree_of_constructable_points}
    Let $\mathcal{M}\subseteq\IC$ with $0,1 \in \mathcal{M}$ and $K_0 := \IQ(\mathcal{M}\cup \overline{\mathcal{M}})$.
    Then for $z \in \mathcal{M}_{\infty}$ is equvalent:
    \begin{enumerate}
        \item $z \in \mathcal{M}_{\infty}$
        \item There is an $n \in \IN$ and a chain \begin{equation*}
            K_0 = L_0 \subset L_1 \subset \dots \subset L_n \subset \IC
        \end{equation*}
        of subfields of $\IC$ such that $z \in L_n$ and $[L_i:L_{i-1}] =2$ for $1\le i\le n$ %$L_{i+1}$ is obtained from $L_i$ by adjoining a square root.
    \end{enumerate}
    In this case $[K_0(z):K_0] = 2^m$ for some $0 \le m \le n$.
\end{theorem}
\begin{remark}
    Theorem \ref{thm:degree_of_constructable_points} tells us that it is sufficient to show that $[K_0(z):K_0] \ne 2^m$ for some $0 \le m $, witch we will use to show that $\sqrt[3]{2} \notin \mathcal{M}_{\infty}$ and $\exp(\textbf{i} \alpha/3) \notin \mathcal{M}_{\infty}$ for some $\alpha$.
\end{remark}
\begin{proof}[Proof of Theorem \ref{thm:degree_of_constructable_points}]
    TODO %TODO
\end{proof}

\subsection{Doubling the cube with a compass and straightedge}
The Problem of doubling the cube is equivalent to the question if $\sqrt[3]{2} \in \mathcal{M}_{\infty}$.
Since $\mathcal{M} = \{0,1\}$, we know that $K_0 = \IQ$. Therfore we need examine if $[\IQ(\sqrt[3]{2}):\IQ] = 2^m$.

\begin{theorem}
    $P := X^3 - 2$ is irreducible over $\IQ$.
\label{thm:irreducible_over_Q}
\end{theorem}
\begin{proof}
    Since $\IQ$ is $a$ subfield of $\IC [X]$, we know that
    \begin{equation*}
        X^3 - 2 = (X - \sqrt[3]{2})(X -\zeta_3 \sqrt[3]{2})(X -\zeta_3^2 \sqrt[3]{2})
    \end{equation*}
    Suppose $P$ is Rational, then
    \begin{equation*}
        X^3 - 2 = (X - a)(X^2 + bX + c)\text{, with } a, b, c \in \IQ
    \end{equation*}
    In particular it has a zero in $\IQ$, so there is a rational number $a$ such that $a^3 = 2$.\newline
    But we know that $\zeta_3 \sqrt[3]{2}$ and $\zeta_3^2 \sqrt[3]{2}$ are not real numberns and $\sqrt[3]{2}$ is not rational.
    So $P$ is irreducible over $\IQ$.
\end{proof}
\begin{theorem}
    The cube can't be doubled using a compass and straightedge.
\end{theorem}
\begin{proof}
    We know that $K_0 = \IQ$ and the problem is equivalent to $\sqrt[3]{2} \in \mathcal{M}_{\infty}$.\newline
    \ref{thm:degree_of_constructable_points} tells us that it is sufficient to show that $[\IQ(\sqrt[3]{2}):\IQ] \ne 2^m$ for some $0 \le m $.
    \newline
    We know that $P := X^3 - 2$ is irreducible over $\IQ$ \ref{thm:irreducible_over_Q} and $P(\sqrt[3]{2}) = 0$, therefore $P = \mu_{\sqrt[3]{2},\IQ}$.
    So with \ref{thm:degree_of_simple_field_extension} we know $[\IQ(\sqrt[3]{2}):\IQ] = 3$ and $[\IQ(\sqrt[3]{2}):\IQ] \ne 2^m$ for some $0 \le m $.
\end{proof}


\subsection{Angle trisection}
Let $\mathcal{M} = \{0,1,\exp(\textbf{i} \alpha)\}$ with $\alpha \in (0,2\pi)$. Therfore we know that
\begin{equation*}
    K_0 = \IQ(\mathcal{M}\cup \overline{\mathcal{M}}) = \IQ(\exp(\textbf{i} \alpha))
\end{equation*}
We need to examine if $\exp(\textbf{i} \alpha/3) \in \mathcal{M}_{\infty}$. Since \ref{thm:degree_of_constructable_points}
that for an postive answer it is nessary that $[\IQ(\exp(\textbf{i} \alpha/3)):\IQ] = 2^m$ for some $0 \le m $. \newline
Since $\exp(\textbf{i} \alpha/3)$ is zero of $X^3 - \exp(\textbf{i} \alpha)$, we know that $[\IQ(\exp(\textbf{i} \alpha/3)):\IQ] \le 3$.
Therfore it is equivalent
\begin{enumerate}
    \item $\exp(\textbf{i} \alpha/3) \notin \mathcal{M}_{\infty}$
    \item $\text{degree}(\mu_{\exp(\textbf{i} \alpha/3),\IQ}) = 3$
    \item $X^3 - \exp(\textbf{i} \alpha/3)$ is irreducible over $\IQ$
\end{enumerate}

\begin{theorem}
    The angle $\pi / 3 = 60^{\circ}$ can't be trisected using a compass and straightedge.
\end{theorem}
\begin{proof}
    We know
    \begin{equation*}
        \exp(\textbf{i}x) = \cos(x) + \textbf{i} \sin(x) \quad \forall x \in \IR
    \end{equation*}
    For $\alpha = \pi / 3$ we get
    \begin{equation*}
        \cos(\alpha) = \frac{1}{2}\qquad \text{and}\qquad \sin(\alpha) = \frac{\sqrt{3}}{2}
    \end{equation*}
    Since we know that $\sqrt{r} \in \mathcal{M}_{\infty}$ for $r \in \mathcal{M}_n$
    we see that $\exp(\textbf{i} \alpha) \in \mathcal{M}_{\infty}$ for $\mathcal{M} = \{0,1\}$. \newline
    So we will work with $K_0 = \IQ$. \newline
    We also now that if $x\in \mathcal{M}_{\infty}$, then $x.real, x.imag \in \mathcal{M}_{\infty}$. So we focus on $\cos(\alpha/3)$, witch is zero of
    \begin{equation*}
        f := 8 X^3 - 6 X - 1 \in \IQ[X]
    \end{equation*}
    Suppose $f$ is reducible over $\IQ$, then $f$ has a rational zero $a$, since $f$ is of degree $3$. According to the rational root theorem, a root rational root of $f$ is of the form $\pm \frac{p}{q}$ with $p$ a factor of the constant term and $q$ a factor of the leading coefficient. So the only possible rational zeros of $f$ are
     \begin{equation*}
        \{ \pm 1, \pm \frac{1}{2}, \pm \frac{1}{4}, \pm \frac{1}{8} \}.
     \end{equation*}
     One can check that none of these numbers is a zero of $f$.
     So $f$ is irreducible over $\IQ$ and $\cos(\alpha/3) \notin \mathcal{M}_{\infty}$.
     Therefore
        \begin{equation*}
            \exp(\textbf{i} \alpha/3) \notin \mathcal{M}_{\infty}
        \end{equation*}
    So the angle $\pi / 3 = 60^{\circ}$ can't be trisected using a compass and straightedge.
\end{proof}

\section {Proof in Lean}
\subsection{Definitions in Lean}
We have priviously defined the set of constructable points over lines and circles. We will now define the set of constructable points over the complex numbers.
\begin{definition}
    Let \textbf{Point} be an $z\in\IC$. For points $z_i$ we write $x_i$ and $y_i$ for $z_i = x_i +\mathbf{i}y_i$.
\end{definition}
\begin{definition}
    Let \textbf{$\mathcal{G}$} be a set of Points depending on Point $z_1$ and $z_2$.
    \begin{equation*}
        \mathcal{G}_{z_1 ,z_2} := \{\lambda z_1 + (1 - \lambda) z_2 | \lambda \in \IR  \}
    \end{equation*}
    Let \textbf{$\mathcal{C}$} be a set of Points depending on Point $z_1$ and $r:=\|z_2-z_3\|$.
    \begin{equation*}
        \begin{aligned}
            \mathcal{C}_{z_1,r} :=& \{ x +\textbf{i} y\in \IC | (x - x_1)^2 - (\textbf{i} y-\textbf{i} y_1)^2 =r^2  \ \}\\
            =& \{ x +\textbf{i} y\in \IC | (x - x_1)^2 + (y-y_1)^2 =r^2  \ \}
        \end{aligned}
        \end{equation*}
\end{definition}

\begin{definition}
The rules (ZL1), (ZL2) and (ZL3) define the Sets of Points:
\begin{enumerate}
    \item Z\_one\_M $\{z\in \IC | \exists z_1,\dots,z_4\in\mathcal{M}: z \in \mathcal{G}_{z_1,z_2} \cap \mathcal{G}_{z_3,z_4} \text{ and } z_3 \ne z_1 \ne z_4\}$
    \item Z\_two\_M $\{z\in \IC | \exists z_1,\dots,z_5\in\mathcal{M}: z \in \mathcal{G}_{z_1,z_2} \cap \mathcal{C}_{z_3,\|z_4-z_5\|} \text{ and } z_4\ne z_5\}$
    \item Z\_three\_M $\{z\in \IC | \exists z_1,\dots,z_6\in\mathcal{M}: z \in \mathcal{C}_{z_1,\|z_2-z_3\|}  \cap \mathcal{C}_{z_4,\|z_5-z_6\|} \text{ and } z_1\ne z_4, z_2\ne z_3,z_5\ne z_6\}$
\end{enumerate}
Therefore $\mathcal{Z(M)}$ is definde as $ \mathcal{Z(M)}:= \mathcal{M} \cup \text{Z\_one\_M}\cup\text{Z\_two\_M}\cup\text{Z\_three\_M}$.
\end{definition}

\begin{definition}
    We define inductively the the chain
    \begin{equation*}
         M_I :  \IN \mapsto \mathcal{M} := \begin{cases}
            \mathcal{M} & \text{if } n = 0 \\
            \mathcal{Z}(M_I(n-1)) & \text{if } n > 0
        \end{cases}
    \end{equation*}
    And the set of all constructable points as $\mathcal{M}_{\infty} = \bigcup_{n \in \IN} \mathcal{M}_n$.
\end{definition}

\section{Conclusion on proofing in Lean}
Lean is great for proving general theorems, but it is not the best tool for proving specific problems.
Proving construction problems with ruler and compass often requires working with complex numbers and field extensions or explicit polynomials.
These are not computable in Lean. Therefore, proving these problems in Lean is a very long process. The witch can be seen in endless lines of commented out code. \newline
But Zulip was a great help. The community is very helpful and the response time is very short.
\subsection{What would help to prove these problems in Lean}
It would be helpful if polynomial and field extensions were computable in Lean. Also, it would be helpful if there were more lemmas and theorems about complex numbers, because there are few lemma witch are helpful, but are only shown for Real numbers.  \newline
Also, it would be helpful if there were Tatict for computing explicit goals, like \footnote{https://leanprover.zulipchat.com/#narrow/stream/113489-new-members/topic/.E2.9C.94.20Eval.20of.20polynomial/near/420729683}. \newline
And an overview of the existing `Taticts`' or somthing like `Tatict?` witch shows you all the Taticts that are available and could be usful for the current goal.

\subsection{What still needs to be done}
Tidying up the code and making it more readable would help a lot. Also lemma and theorems that are used can be generalised and statet outside of the specific problem or proof. So that they can be reused in other proofs. \newline
Also there are some lemmas that are not yet proven:
\subsubsection*{Real Component In M Inf}
If $z \in \mathcal{M}_{\infty}$, then $z.real \in \mathcal{M}_{\infty}$.
Witch can be proven by using the line trough $z$ and $\overline{z}$ and the fact and the fact that you can halve distances. Or by using $real(z)= z+ \overline{z}$.
\subsubsection*{Classfication Z In M Inf}
The Formalisation of the construction problems with ruler and compass is not yet complete.
The proof of Theorem \ref{thm:degree_of_constructable_points} and the underlying lemmma over $\mathcal{M}_{\infty}$ are still missing.
Here we need to describe differnt Points as soloitions and painstakingly prove that they are not in $\mathcal{M}_{\infty}$.
\subsubsection*{Classfication Z In M Inf 2m}
To prove that $z \in \mathcal{M}_{\infty}$ implies $[K_0(z):K_0] = 2^m$ for some $0 \le m \le n$. We use \ref{thm:degree_of_simple_field_extension}, \ref{thm:degree_of_constructable_points} and the degreeformular for field extensions.
\subsubsection*{Pi Third Not In M Inf}
To finsih the proof it is nessary to show why it is sufficient to use the polynomial $f := 8 X^3 - 6 X - 1$ over $\IQ$. Witch is esay to stat by using $\exp(\textbf{i}x) = \cos(x) + \textbf{i} \sin(x) \quad \forall x \in \IR$, $\cos(\pi/3) = \frac{1}{2}$, $\sin(\pi/3) = \frac{\sqrt{3}}{2}$ and the fact that $\sqrt{r} \in \mathcal{M}_{\infty}$ for $r \in \mathcal{M}_n$.
But since you can't compute $\mathcal K_{\inf}$ or $\mathcal M_{\inf}$ in Lean, it is hard prove that $M_{\inf}= \IQ_{\inf}$ so one can use the minimal polynomial of $\cos(\pi/3)$ over $\IQ$.

\end{document}